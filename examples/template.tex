%% start of file `template.tex'.
%% Copyright 2006-2015 Xavier Danaux (xdanaux@gmail.com).
%
% This work may be distributed and/or modified under the
% conditions of the LaTeX Project Public License version 1.3c,
% available at http://www.latex-project.org/lppl/.


\documentclass[11pt,a4paper,sans]{moderncv}        % possible options include font size ('10pt', '11pt' and '12pt'), paper size ('a4paper', 'letterpaper', 'a5paper', 'legalpaper', 'executivepaper' and 'landscape') and font family ('sans' and 'roman')

% moderncv themes
\moderncvstyle{classic}                             % style options are 'casual' (default), 'classic', 'banking', 'oldstyle' and 'fancy'
\moderncvcolor{burgundy}                               % color options 'black', 'blue' (default), 'burgundy', 'green', 'grey', 'orange', 'purple' and 'red'
%\renewcommand{\familydefault}{\sfdefault}         % to set the default font; use '\sfdefault' for the default sans serif font, '\rmdefault' for the default roman one, or any tex font name
%\nopagenumbers{}                                  % uncomment to suppress automatic page numbering for CVs longer than one page

% character encoding
%\usepackage[utf8]{inputenc}                       % if you are not using xelatex ou lualatex, replace by the encoding you are using
%\usepackage{CJKutf8}                              % if you need to use CJK to typeset your resume in Chinese, Japanese or Korean

% adjust the page margins
\usepackage[scale=0.8]{geometry}
%\setlength{\hintscolumnwidth}{3cm}                % if you want to change the width of the column with the dates
%\setlength{\makecvheadnamewidth}{10cm}            % for the 'classic' style, if you want to force the width allocated to your name and avoid line breaks. be careful though, the length is normally calculated to avoid any overlap with your personal info; use this at your own typographical risks...

% personal data
\name{Yathirajan}{Manavalan}
% \title{Resumé title}                               % optional, remove / comment the line if not wanted
\address{Toronto, Ontario}{M5R2X3}{}% optional, remove / comment the line if not wanted; the "postcode city" and "country" arguments can be omitted or provided empty
\phone[mobile]{+1~(647)~575~0351}                   % optional, remove / comment the line if not wanted; the optional "type" of the phone can be "mobile" (default), "fixed" or "fax"
\email{yathi@outlook.com}                               % optional, remove / comment the line if not wanted
% \homepage{www.johndoe.com}                         % optional, remove / comment the line if not wanted
\social[github]{Yathi}                              % optional, remove / comment the line if not wanted
\social[linkedin]{yathirajan-manavalan-06a68a19}                        % optional, remove / comment the line if not wanted
% \social[xing]{john\_doe}                           % optional, remove / comment the line if not wanted
\social[twitter]{ihtay}                             % optional, remove / comment the line if not wanted
% \social[gitlab]{jdoe}                              % optional, remove / comment the line if not wanted
% \social[skype]{jdoe}                               % optional, remove / comment the line if not wanted
% \extrainfo{additional information}                 % optional, remove / comment the line if not wanted
% \photo[64pt][0.4pt]{picture}                       % optional, remove / comment the line if not wanted; '64pt' is the height the picture must be resized to, 0.4pt is the thickness of the frame around it (put it to 0pt for no frame) and 'picture' is the name of the picture file
% \quote{Some quote}                                 % optional, remove / comment the line if not wanted

% bibliography adjustements (only useful if you make citations in your resume, or print a list of publications using BibTeX)
%   to show numerical labels in the bibliography (default is to show no labels)
%\makeatletter\renewcommand*{\bibliographyitemlabel}{\@biblabel{\arabic{enumiv}}}\makeatother
\renewcommand*{\bibliographyitemlabel}{[\arabic{enumiv}]}
%   to redefine the bibliography heading string ("Publications")
%\renewcommand{\refname}{Articles}

% bibliography with mutiple entries
%\usepackage{multibib}
%\newcites{book,misc}{{Books},{Others}}
%----------------------------------------------------------------------------------
%            content
%----------------------------------------------------------------------------------
\begin{document}
%-----       resume       ---------------------------------------------------------
\makecvtitle

%----------------------------------------------------------------------------------------
%	Intro SECTION
%----------------------------------------------------------------------------------------

\section{Professional Summary}

\cvitem{}{A software developer with 9+ years of experience building performant and robust web applications with insight into the loyalty and finance industries and payment systems. Vim and Dvorak user and a lifelong learner.
\newline{}
\begin{itemize}
\item Ruby on Rails, Typescript, React, EmberJS
\item Enthusiastic about functional programming paradigms and dabbled with Haskell and Elixir. Currently learning Rust.
\item Strong knowledge of OO principles and design patterns
\item Extensive troubleshooting and debugging skills
\item Excellent mentorship and leadership skills
\end{itemize}
}

%----------------------------------------------------------------------------------------
%	WORK EXPERIENCE SECTION
%----------------------------------------------------------------------------------------

\section{Experience}

% \cventry{year--year}{Job title}{Employer}{City}{}{General description no longer than 1--2 lines.\newline{}%
\cventry{Aug-2023 -- Present}{Developer}{\textsc{Instacart}}{Toronto}{}{
\begin{itemize}
\item Taking over ownership of third party integrations service within the Professional Services Unit.
  \begin{itemize}
    \item As part of it, I have been standardizing the various plugin implementations and the returned error responses.
    \item Improving the test coverage as well adding tests for edge cases missed.
    \item Creating and improving our datadog monitors.
    \item Improving the usage of Sorbet within the various plugins and working on improving the availability of sorbet types from other services outside of 3PI.
  \end{itemize}
\item Working closely with the loyalty team and business development to launch multiple retailer integrations and various retailer promotions.
\item Working with Data Science and running experiments to see how the integrations and the promotions are performing.
\end{itemize}}
\bigskip
\cvitem{}{\textbf{Technologies Used: }Ruby on Rails, Sorbet, gRPC}
\bigskip

\cventry{May-2022 -- May-2023}{Developer}{\textsc{Shopify}}{Remote}{}{
\begin{itemize}
\item Built integrations to services in different countries to verify merchant identity and detect fraudulent ones. 
\item Worked with Shopify Payments to build consumers for stripe webhooks. 
\item Improved type safety within the merchant profile systems and wrote Tapioca DSL compilers to generate type signatures for internal gems that did not have type signatures.
\end{itemize}}
\bigskip
\cvitem{}{\textbf{Technologies Used: }Ruby on Rails, Sorbet, GraphQL}

%------------------------------------------------
\bigskip
\cventry{Mar-2016 -- Apr-2022}{Senior Full Stack Developer}{\textsc{RewardOps}}{Toronto}{ON}{
\begin{itemize}
\item Architected and implemented numerous features for the Reward management system used by program managers at RewardOps and its clients
\item Was an integral part of various program launches including Aeroplan, Airmiles and US Bank Rewards and implemented an asynchronous event driven order placement flow which considerably sped up the order creation speeds and API response times.
\item Rearchitected and improved the order refund and cancellation system using event-sourcing techniques
\item Implemented a filter string parser as an open-source JS library. This library is capable of recursively parsing filter definitions with different combinations of conditions and it also has methods to update the filter strings. ({https://github.com/rewardops/javascript-filter-parser})
\item Optimized the memory efficiency of the reporting system which drastically reduced the memory load on the servers
\item Took charge in conducting lunch and learns and improving documentation
\end{itemize}}
\bigskip
\cvitem{}{\textbf{Technologies Used: }Ruby on Rails, React, EmberJS}

%------------------------------------------------

% \bigskip
% \cventry{2010--2013}{Java Developer}{\textsc{Tata Consultancy Services}}{Chennai}{}{
%   \begin{itemize}
%     \item Built the MoneyGram-PayPal integration
%     \item Was part of a major technology platform upgrade within MoneyGram
%     \item Interacted with various clients and guided them through the upgrade process
%   \end{itemize}
% }
% \bigskip
% \cvitem{}{\textbf{Technologies Used: }Java, Javascript, jQuery, SVN}

% %------------------------------------------------
% \newpage

%----------------------------------------------------------------------------------------
%	EDUCATION SECTION
%----------------------------------------------------------------------------------------

\section{Education}

% \cventry{year--year}{Degree}{Institution}{City}{\textit{Grade}}{Description}  % arguments 3 to 6 can be left empty
\cventry{2013--2015}{Master of Science, Computing Science}{University of Alberta}{Edmonton}{}{}  % Arguments not required can be left empty
\cventry{2006--2010}{Bachelor of Engineering, Electronics and Communication Engineering}{Anna University}{Chennai}{}{}

\section{Masters Thesis}

\cvitem{Title}{\emph{A Light-weight AI emotion model for NPCs in a video game}}
\cvitem{Supervisors}{Professor Vadim Bulitko}
\cvitem{Description}{This thesis explored the idea that basing NPC actions on a simulated emotion model based on fear, greed, happiness and sorrow would lead to more organic NPC reactions to player actions in a video game.}

\section{Interests}
\cvitem{Photography}{https://500px.com/p/thunderemperor}
\cvitem{Video Games}{Regular: DotA. Currently playing Elden Ring and Hollow Knight}
\cvitem{NeoVim}{Love tinkering with it and trying to get better.}

% \section{Extra 1}
% \cvlistitem{Item 1}
% \cvlistitem{Item 2}
% \cvlistitem{Item 3. This item is particularly long and therefore normally spans over several lines. Did you notice the indentation when the line wraps?}

% \section{Extra 2}
% \cvlistdoubleitem{Item 1}{Item 4}
% \cvlistdoubleitem{Item 2}{Item 5\cite{book1}}
% \cvlistdoubleitem{Item 3}{Item 6. Like item 3 in the single column list before, this item is particularly long to wrap over several lines.}

% \section{References}
% \begin{cvcolumns}
%   \cvcolumn{Category 1}{\begin{itemize}\item Person 1\item Person 2\item Person 3\end{itemize}}
%   \cvcolumn{Category 2}{Amongst others:\begin{itemize}\item Person 1, and\item Person 2\end{itemize}(more upon request)}
%   \cvcolumn[0.5]{All the rest \& some more}{\textit{That} person, and \textbf{those} also (all available upon request).}
% \end{cvcolumns}

% Publications from a BibTeX file without multibib
%  for numerical labels: \renewcommand{\bibliographyitemlabel}{\@biblabel{\arabic{enumiv}}}% CONSIDER MERGING WITH PREAMBLE PART
%  to redefine the heading string ("Publications"): \renewcommand{\refname}{Articles}
\nocite{*}
\bibliographystyle{plain}
\bibliography{publications}                        % 'publications' is the name of a BibTeX file

% Publications from a BibTeX file using the multibib package
%\section{Publications}
%\nocitebook{book1,book2}
%\bibliographystylebook{plain}
%\bibliographybook{publications}                   % 'publications' is the name of a BibTeX file
%\nocitemisc{misc1,misc2,misc3}
%\bibliographystylemisc{plain}
%\bibliographymisc{publications}                   % 'publications' is the name of a BibTeX file

\clearpage
%-----       letter       ---------------------------------------------------------
% recipient data
% \recipient{Company Recruitment team}{Company, Inc.\\123 somestreet\\some city}
% \date{January 01, 1984}
% \opening{Dear Sir or Madam,}
% \closing{Yours faithfully,}
% \enclosure[Attached]{curriculum vit\ae{}}          % use an optional argument to use a string other than "Enclosure", or redefine \enclname
% \makelettertitle

% Lorem ipsum dolor sit amet, consectetur adipiscing elit. Duis ullamcorper neque sit amet lectus facilisis sed luctus nisl iaculis. Vivamus at neque arcu, sed tempor quam. Curabitur pharetra tincidunt tincidunt. Morbi volutpat feugiat mauris, quis tempor neque vehicula volutpat. Duis tristique justo vel massa fermentum accumsan. Mauris ante elit, feugiat vestibulum tempor eget, eleifend ac ipsum. Donec scelerisque lobortis ipsum eu vestibulum. Pellentesque vel massa at felis accumsan rhoncus.

% Suspendisse commodo, massa eu congue tincidunt, elit mauris pellentesque orci, cursus tempor odio nisl euismod augue. Aliquam adipiscing nibh ut odio sodales et pulvinar tortor laoreet. Mauris a accumsan ligula. Class aptent taciti sociosqu ad litora torquent per conubia nostra, per inceptos himenaeos. Suspendisse vulputate sem vehicula ipsum varius nec tempus dui dapibus. Phasellus et est urna, ut auctor erat. Sed tincidunt odio id odio aliquam mattis. Donec sapien nulla, feugiat eget adipiscing sit amet, lacinia ut dolor. Phasellus tincidunt, leo a fringilla consectetur, felis diam aliquam urna, vitae aliquet lectus orci nec velit. Vivamus dapibus varius blandit.

% Duis sit amet magna ante, at sodales diam. Aenean consectetur porta risus et sagittis. Ut interdum, enim varius pellentesque tincidunt, magna libero sodales tortor, ut fermentum nunc metus a ante. Vivamus odio leo, tincidunt eu luctus ut, sollicitudin sit amet metus. Nunc sed orci lectus. Ut sodales magna sed velit volutpat sit amet pulvinar diam venenatis.

% Albert Einstein discovered that $e=mc^2$ in 1905.

% \[ e=\lim_{n \to \infty} \left(1+\frac{1}{n}\right)^n \]

% \makeletterclosing

%\clearpage\end{CJK*}                              % if you are typesetting your resume in Chinese using CJK; the \clearpage is required for fancyhdr to work correctly with CJK, though it kills the page numbering by making \lastpage undefined
\end{document}


%% end of file `template.tex'.
